\documentclass[11pt, a4paper]{article}
\usepackage{ctex}
\usepackage[body={14.64cm, 24.62cm}, centering, dvipdfm]{geometry}

\usepackage{fancyhdr}
\usepackage{listings}
\usepackage{color}
\usepackage{fontspec}
\usepackage{graphicx}
\usepackage{float}
\usepackage{extramarks}
\usepackage{amsmath}
\setmainfont{Times New Roman}
\setmonofont{Consolas}
\usepackage{tikz}
\usepackage{wallpaper}
\newcommand\BackgroundPicture{%
	\put(0,0){%
		\parbox[b][\paperheight]{\paperwidth}{%
			\vfill
			\centering%
			\begin{tikzpicture}[remember picture,overlay]
			\node [rotate=0,scale=1.1,text opacity=0.075] at (current page.center)
			{ 
				\includegraphics{SJTU_BLUE.png}       %\copyright XXX Powered by~\LaTeX
			};
			\end{tikzpicture}%
			%      \vfill
		}}}
		
\definecolor{mygreen}{rgb}{0,0.6,0}
\definecolor{mygray}{rgb}{0.5,0.5,0.5}
\definecolor{mymauve}{rgb}{0.58,0,0.82}

\newfontface\YaHeiMono{Microsoft YaHei Mono}
\newfontface\newSimSum{新宋体}

\lstdefinelanguage{JavaScript}{
	keywords={typeof, new, true, false, catch, function, return, null, catch, switch, var, if, in, while, do, else, case, break},
	keywordstyle=\color{blue}\bfseries,
	ndkeywords={class, export, boolean, throw, implements, import, this},
	ndkeywordstyle=\color{darkgray}\bfseries,
	identifierstyle=\color{black},
	sensitive=false,
	comment=[l]{//},
	morecomment=[s]{/*}{*/},
	commentstyle=\color{purple}\ttfamily,
	stringstyle=\color{red}\ttfamily,
	morestring=[b]',
	morestring=[b]"
}

\lstset{ %
	backgroundcolor=\color{white},   % choose the background color; you must add \usepackage{color} or \usepackage{xcolor}
	basicstyle=\scriptsize \YaHeiMono,        % the size of the fonts that are used for the code
	breakatwhitespace=true,         % sets if automatic breaks should only happen at whitespace
	breaklines=true,                 % sets automatic line breaking
	captionpos=none,                    % sets the caption-position to bottom
	commentstyle=\color{mygreen} \newSimSum,
	morecomment=[l][\color{magenta}]{\#},    % comment style
	deletekeywords={...},            % if you want to delete keywords from the given language
	escapeinside={\%*}{*)},          % if you want to add LaTeX within your code
	extendedchars=true,              % lets you use non-ASCII characters; for 8-bits encodings only, does not work with UTF-8
	framexleftmargin=0.5em,
	frame=single,                    % adds a frame around the code
	keepspaces=true,                 % keeps spaces in text, useful for keeping indentation of code (possibly needs columns=flexible)
	keywordstyle=\color{blue},       % keyword style
	language=C++,                 % the language of the code
	otherkeywords={*,...},            % if you want to add more keywords to the set
	numbers=none,                    % where to put the line-numbers; possible values are (none, left, right)
	numbersep=1.5em,                   % how far the line-numbers are from the code
	numberstyle=\tiny\color{mygray}, % the style that is used for the line-numbers
	rulecolor=\color{black},         % if not set, the frame-color may be changed on line-breaks within not-black text (e.g. comments (green here))
	showspaces=false,                % show spaces everywhere adding particular underscores; it overrides 'showstringspaces'
	showstringspaces=false,          % underline spaces within strings only
	showtabs=false,                  % show tabs within strings adding particular underscores
	stepnumber=1,                    % the step between two line-numbers. If it's 1, each line will be numbered
	stringstyle=\color{red},     % string literal style
	tabsize=4,                       % sets default tabsize to 2 spaces
	xleftmargin=2em,xrightmargin=2em  
}

\title{TinyScheme——一个微型Scheme语言解释器}
\author{柯嵩宇 \\ 上海交通大学}

\begin{document}
	\AddToShipoutPicture{\BackgroundPicture}
	\maketitle
	\tableofcontents
	\newpage
	
	\section{简介}	
	一个实现了基本Scheme语言语法的解释器。
	\section{运行环境}
		解释器用Javascript编写,运行环境为node.js(v0.12.7)。使用了开源的高精度库number-crunch实现对大整数运算的支持。项目目录中已经包含了该模块(./node\_modules/number-crunch),若没有该模块可以执行:\verb|npm install number-crunch|获取。\footnote{通过npm方式获取的number-crunch库有一个小bug——不能够将内部数据类型的负数转换成字符串形式,我在GitHub上与源代码一起提供了我修改后的number-crunch库。}
		
		执行的时候保证main.js,util.js以及包含number-crunch库的node\_modules目录在一个文件夹下,在终端中执行命令“node ./main.js”。解释器会从"./src.scm"中读取Scheme并解释执行,同时将执行结果输出到标准输出(stdout)上。
		\subsection{number-crunch库简介}
			用Javascript编写的开源高精度库,用28位整数数组来实现对大整数的存储和运算,提供了将字符串表示的大整数转换为运算所需的整数数组以及将结果重新转换成字符串表示的函数。
			
			支持的大整数操作包括:
			\begin{itemize}
				\item 加、减、乘、除、取模
				\item 大整数开平方根(向零取整)
				\item 乘法逆元等常用的整数运算算法。
			\end{itemize}
	\section{项目的实现}
		\subsection{运行细节}
			解释器启动时先进行一系列的初始化操作后会从"./src.scm"中读取Scheme语言语句,然后通过util.GetSentents函数将源代码文件分割为单独的语句并且逐句执行。
			
			通过ProcExec函数执行已划分好的每一条语句。
			
			ProcExec通过调用util.GetElements进行语句的分词,获取每一个参数的类型,然后通过ProcessParas处理参数,获得参数的实际值,然后根据被调用的过程来执行不同的处理操作。
		\subsection{语法和函数}
			在编写代码的时候,我把过程分为语法(syntax)和函数(function)两类,两者的区别是,函数有自己的域(scope),管理函数内定义的局部变量,而语法没有,执行语法时始终从当前可见域中获取数据。由于Javascript对于对象的赋值都是地址赋值,所以在传递参数的时候加上可见域不会对效率造成致命的影响。\footnote{函数定义时会记录当前的可见域,同时在函数调用的时候会对该函数的可见域进行一层拷贝(不需要深度拷贝),因此,在大量递归的时的表现并不是很好,而且由于Javascript栈空间不足,N皇后问题最多只能支持到7皇后。}
			
			在Scheme语法标准中,Scheme的解释器必须对尾递归进行优化,但是由于我不能找到正确判断是否为尾递归的方法,因此就没有对函数尾递归进行优化。
		\subsection{数据的存储}
			Scheme语言本身是弱类型的,但是在运算时不同的数据类型会有不一样的表现。同时考虑到Javascript同样也是弱类型语言,解释器采用了下面的object来保存数据:
\begin{lstlisting}[language=Javascript]
{
	"type" : ***,
	***
}
\end{lstlisting}
		\subsection{支持的数据类型}
			\begin{enumerate}
				\item integer
				\item float
				\item char
				\item string
				\item pair
				\item list
				\item symbol
				\item function
			\end{enumerate}
		\subsection{函数的实现细节}
			采用了下面的结构来保存函数的信息:
\begin{lstlisting}[language=Javascript]
{
	"type" : "function",
	"paraList" : [...],
	"body" : [...],
	"scope" : {...},
}
\end{lstlisting}
			\begin{description}
				\item[type] 表示这个对象的类型是一个函数。
				\item[paraList] 函数的形参列表。
				\item[body] 保存函数体的每一条语句。
				\item[scope] 保存函数执行前的可见域。\footnote{用于实现Lexcial Scope}
			\end{description}
			
			函数在执行时会对scope执行一次复制(浅复制),然后以副本作为当前的可见域执行语句。
		\subsection{一些等价的转换}
			\subsubsection{let}
				在实现中,把let环境转换为lambda表达式执行,如:
\begin{lstlisting}[language=lisp]
(let ((var1 expr1)
      (var2 expr2)
      ...
      (varN exprN))
     body1
     ...
     bodyN)
\end{lstlisting}
转换为:
\begin{lstlisting}[language=lisp]
((lambda (var1 var2 ... varN)
	body1
	...
	bodyN)
	expr1 expr2 ... exprN)
\end{lstlisting}

			\subsubsection{let*}
				在实现时,let*环境被视为let环境的嵌套,如:
\begin{lstlisting}[language=lisp]
(let* ((var1 expr1)
       (var2 expr2))
      body1
      body2)
\end{lstlisting}
转换为:
\begin{lstlisting}[language=lisp]
(let ((var1 expr1))
	(let ((var2 expr2))
		body1
		body2))
\end{lstlisting}

			\subsubsection{letrec}
				letrec环境被lambda语句重新封装,如:
\begin{lstlisting}[language=lisp]
(letrec ((<var1> <expr1>)
         (<var2> <expr2>))
		body1
		body2)
\end{lstlisting}
转换为:
\begin{lstlisting}[language=lisp]
((lambda ()
	(define <var1>.1 expr1)
	(define <var2>.1 expr2)
	(define <var1> <var1>.1)
	(define <var2> <var2>.1)
	body1
	body2))
\end{lstlisting}

	\section{实现的语法}
		\subsection{基本语法}
			\begin{itemize}
				\item define
				\item lambda
				\item map
				\item apply
				\item if
				\item cond
				\item begin
				\item quote
				\item display
				\item newline
				\item let,let*,letrec
			\end{itemize}
		\subsection{数值运算相关}
			\begin{itemize}
				\item $+,-,*,/$
				\item quotient
				\item modulo
				\item sqr
			\end{itemize}
		\subsection{逻辑运算相关}
			\begin{itemize}
				\item and
				\item or
				\item not
			\end{itemize}
		\subsection{list和pair相关}
			\begin{itemize}
				\item list
				\item cons
				\item car
				\item cdr
				\item length
				\item append
				\item reverse
				\item list-ref
				\item memq
				\item assq
			\end{itemize}
		\subsection{一元断言与二元断言}
			\begin{itemize}
				\item eq?\footnote{该断言可能存在一些问题,在MIT Scheme的参考手册中,eq?的存在大量未定义的行为。}
				\item eqv?
				\item equal?
				\item $<,>,=,\leq$
				\item zero?
				\item string=?
				\item null?
				\item pair?
				\item list?
			\end{itemize}
	\section{总结}
		\subsection{过程中的收获}
			开始的时候由于考虑用什么语言来完成纠结了很久。先后考虑了C++、Python、Haskell和Javascript,但是由于我之前并不会这两门语言,所以用了一个星期的时间来学习。最后基于以下几点选择了Javascript:
			\begin{enumerate}
				\item Scheme是弱类型语言,而Javascript也是弱类型语言,这样可以避免用强类型语言来模拟弱类型语言时复杂的抽象过程。
				\item Javascript的执行效率令人满意,不像Python的效率那么令人堪忧。
				\item Javascript的语法与C++相似。
				\item 可以轻松的获得开源的高精度运算库。
			\end{enumerate}	
			
			在完成解释器,积累了大量的经验之余掌握了两门编程语言,这也算是额外的收获吧。
		\subsection{设计上的优点}
			本人对自己的设计还是比较看好的。利用Javascript的语言特性使得我可以很轻松地在现有代码的基础上增加对新的语法的支持而不需要改动大量的代码。\footnote{只要在./main.js中加入对相关语法行为的定义就好了。}
		\subsection{关于语法糖的一些看法}
			由于Scheme本身并不支持迭代,需要通过递归操作完成对list的遍历。在没有对尾递归进行优化的情况下会造成递归引起巨大的内存消耗,从而让程序的执行效率降低。因此,虽然在编写中使用“语法糖”可以减少很多的代码量,但是这会使得许多原本可以用线性迭代模拟的操作因为没有尾递归优化而变成了线性递归的操作,这是让人不能接受的。\footnote{虽然我不支持使用它,但是在实现let环境的时候我还是“偷懒”地用了,因为把它等价转换之后实在是太简单了。}
		\subsection{存在的一些问题}
			由于Javascript的栈空间不足,在大量深度递归的时候有可能会栈溢出。同时由于代码在处理函数递归时的优化较少,出现复杂的函数调用时效率会受到一定的影响。同时变量内存空间的管理完全地托管给Javascript的垃圾回收机制故不适合处理大量数据的情况。
\end{document}
